% !TEX TS-program = pdflatex
% !TEX encoding = UTF-8 Unicode

% This is a simple template for a LaTeX document using the "article" class.
% See "book", "report", "letter" for other types of document.

\documentclass[11pt]{article} % use larger type; default would be 10pt

\usepackage[utf8]{inputenc} % set input encoding (not needed with XeLaTeX)

%%% Examples of Article customizations
% These packages are optional, depending whether you want the features they provide.
% See the LaTeX Companion or other references for full information.

%%% PAGE DIMENSIONS
\usepackage{geometry} % to change the page dimensions
\geometry{a4paper} % or letterpaper (US) or a5paper or....
% \geometry{margin=2in} % for example, change the margins to 2 inches all round
% \geometry{landscape} % set up the page for landscape
%   read geometry.pdf for detailed page layout information

\usepackage{graphicx} % support the \includegraphics command and options

% \usepackage[parfill]{parskip} % Activate to begin paragraphs with an empty line rather than an indent

%%% PACKAGES
\usepackage{listings}
\usepackage{booktabs} % for much better looking tables
\usepackage{array} % for better arrays (eg matrices) in maths
\usepackage{paralist} % very flexible & customisable lists (eg. enumerate/itemize, etc.)
\usepackage{verbatim} % adds environment for commenting out blocks of text & for better verbatim
\usepackage{subfig} % make it possible to include more than one captioned figure/table in a single float
% These packages are all incorporated in the memoir class to one degree or another...

%%% HEADERS & FOOTERS
\usepackage{fancyhdr} % This should be set AFTER setting up the page geometry
\pagestyle{fancy} % options: empty , plain , fancy
\renewcommand{\headrulewidth}{0pt} % customise the layout...
\lhead{}\chead{}\rhead{}
\lfoot{}\cfoot{\thepage}\rfoot{}

%%% SECTION TITLE APPEARANCE
\usepackage{sectsty}
\allsectionsfont{\sffamily\mdseries\upshape} % (See the fntguide.pdf for font help)
% (This matches ConTeXt defaults)

%%% ToC (table of contents) APPEARANCE
\usepackage[nottoc,notlof,notlot]{tocbibind} % Put the bibliography in the ToC
\usepackage[titles,subfigure]{tocloft} % Alter the style of the Table of Contents
\renewcommand{\cftsecfont}{\rmfamily\mdseries\upshape}
\renewcommand{\cftsecpagefont}{\rmfamily\mdseries\upshape} % No bold!

%%% END Article customizations

%%% The "real" document content comes below...

\title{Exercício-Programa 2}
\author{Gustavo Teixeira da Cunha Coelho, Henrique Gemignani Passos Lima}
%\date{} % Activate to display a given date or no date (if empty),
         % otherwise the current date is printed 

\begin{document}
\maketitle

\section{Sobre o relatório}

Esse relatório possui basicamente notas sobre partes específicas do EP, o relatório do grafo NSFnet e instruções de compilação.

\subsection{Sobre a Barreira}
A barreira usada foi a barreira de disseminação, como comentada em aula. Seu código está no arquivo graph.h, das linhas 125 a 133. Ela é executada duas vezes a cada iteração dos threads, para garantir que todos os threads terminem simultaneamente.

\subsection{Relatório NFSNet}
Saída do programa com -debug:
\begin{lstlisting}[breaklines]
Procurando os 2 menores caminhos.
Numero de cores: 16

Iteracao 1:

Iteracao 2:


Saida:
Realizadas 2 iteracoes.
Caminhos para o vertice 1
	0 - 1
	0 - 2 - 1
Caminhos para o vertice 2
	0 - 2
	0 - 1 - 2
Caminhos para o vertice 3
	0 - 1 - 3
	0 - 2 - 1 - 3
Caminhos para o vertice 4
	0 - 4
	0 - 2 - 6 - 5 - 4
Caminhos para o vertice 5
	0 - 2 - 6 - 5
	0 - 1 - 2 - 6 - 5
Caminhos para o vertice 6
	0 - 2 - 6
	0 - 1 - 2 - 6
Caminhos para o vertice 7
	0 - 2 - 6 - 5 - 7
	0 - 1 - 2 - 6 - 5 - 7
Caminhos para o vertice 8
	0 - 1 - 8
	0 - 2 - 1 - 8
Caminhos para o vertice 9
	0 - 2 - 6 - 9
	0 - 1 - 2 - 6 - 9
Caminhos para o vertice 10
	0 - 2 - 6 - 9 - 10
	0 - 1 - 2 - 6 - 9 - 10
Caminhos para o vertice 11
	0 - 2 - 6 - 5 - 4 - 11
	0 - 2 - 6 - 13 - 14 - 11
Caminhos para o vertice 12
	0 - 2 - 6 - 13 - 12
	0 - 1 - 2 - 6 - 13 - 12
Caminhos para o vertice 13
	0 - 2 - 6 - 13
	0 - 1 - 2 - 6 - 13
Caminhos para o vertice 14
	0 - 2 - 6 - 13 - 14
	0 - 1 - 2 - 6 - 13 - 14
Caminhos para o vertice 15
	0 - 2 - 6 - 13 - 14 - 15
	0 - 2 - 6 - 13 - 12 - 15
\end{lstlisting}

\subsection{Compilando}
Para compilar basta executar o comando "cmake . \&\&  make", que rodará o CMake incluso com o EP, gerando assim um arquivo Makefile, que então será executado, gerando o executável.

\end{document}
