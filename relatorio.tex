% !TEX TS-program = pdflatex
% !TEX encoding = UTF-8 Unicode

% This is a simple template for a LaTeX document using the "article" class.
% See "book", "report", "letter" for other types of document.

\documentclass[11pt]{article} % use larger type; default would be 10pt

\usepackage[utf8]{inputenc} % set input encoding (not needed with XeLaTeX)

%%% Examples of Article customizations
% These packages are optional, depending whether you want the features they provide.
% See the LaTeX Companion or other references for full information.

%%% PAGE DIMENSIONS
\usepackage{geometry} % to change the page dimensions
\geometry{a4paper} % or letterpaper (US) or a5paper or....
% \geometry{margin=2in} % for example, change the margins to 2 inches all round
% \geometry{landscape} % set up the page for landscape
%   read geometry.pdf for detailed page layout information

\usepackage{graphicx} % support the \includegraphics command and options

% \usepackage[parfill]{parskip} % Activate to begin paragraphs with an empty line rather than an indent

%%% PACKAGES
\usepackage{listings}
\usepackage{booktabs} % for much better looking tables
\usepackage{array} % for better arrays (eg matrices) in maths
\usepackage{paralist} % very flexible & customisable lists (eg. enumerate/itemize, etc.)
\usepackage{verbatim} % adds environment for commenting out blocks of text & for better verbatim
\usepackage{subfig} % make it possible to include more than one captioned figure/table in a single float
% These packages are all incorporated in the memoir class to one degree or another...

%%% HEADERS & FOOTERS
\usepackage{fancyhdr} % This should be set AFTER setting up the page geometry
\pagestyle{fancy} % options: empty , plain , fancy
\renewcommand{\headrulewidth}{0pt} % customise the layout...
\lhead{}\chead{}\rhead{}
\lfoot{}\cfoot{\thepage}\rfoot{}

%%% SECTION TITLE APPEARANCE
\usepackage{sectsty}
\allsectionsfont{\sffamily\mdseries\upshape} % (See the fntguide.pdf for font help)
% (This matches ConTeXt defaults)

%%% ToC (table of contents) APPEARANCE
\usepackage[nottoc,notlof,notlot]{tocbibind} % Put the bibliography in the ToC
\usepackage[titles,subfigure]{tocloft} % Alter the style of the Table of Contents
\renewcommand{\cftsecfont}{\rmfamily\mdseries\upshape}
\renewcommand{\cftsecpagefont}{\rmfamily\mdseries\upshape} % No bold!

%%% END Article customizations

%%% The "real" document content comes below...

\title{Exercício-Programa 2}
\author{Gustavo Teixeira da Cunha Coelho, Henrique Gemignani Passos Lima}
%\date{} % Activate to display a given date or no date (if empty),
         % otherwise the current date is printed 

\begin{document}
\maketitle

\section{Sobre o relatório}

Esse relatório possui basicamente notas sobre partes específicas do EP, o relatório do grafo NSFnet e instruções de compilação.

\subsection{Sobre a Barreira}
A barreira usada foi a barreira de disseminação, como comentada em aula. Seu código está no arquivo graph.h, das linhas 148 a 156. Ela é executada duas vezes a cada iteração dos threads, para garantir que todos os threads terminem simultaneamente. A chegada na segunda barreira não é impressa com o valor -debug (até mesmo porque a ordem de chegada na segunda barreira é irrelevante comparada com a chegada na primeira). Além disso, temos definido que o ínicio de uma iteração é dado após todos os threads passarem da segunda barreira.

\subsection{Relatório NFSNet}
Notas: o formato da saída com -debug é o seguinte:
\\*- Chegada dos threads na iteração n no formato TN, com N = Número da thread
\\*- Número da Iteração
\\*- Saída dos caminhos para essa iteração.

A máquina usada para a geração deste relatório foi a máquina streetfighter da Rede Linux.
\\*Saída do programa com -debug:
\begin{lstlisting}[breaklines]
Procurando os 2 menores caminhos.
Numero de cores: 16
T0 T1 T2 T3 T4 T5 T6 T7 T8 T9 T10 T11 T12 T13 T14 T15 
Iteracao 1:

Caminhos para o vertice 1.
Caminhos para o vertice 2.
Caminhos para o vertice 3.
Caminhos para o vertice 4.
Caminhos para o vertice 5.
Caminhos para o vertice 6.
Caminhos para o vertice 7.
Caminhos para o vertice 8.
Caminhos para o vertice 9.
Caminhos para o vertice 10.
Caminhos para o vertice 11.
Caminhos para o vertice 12.
Caminhos para o vertice 13.
Caminhos para o vertice 14.
Caminhos para o vertice 15.

T4 T12 T2 T0 T6 T10 T3 T8 T13 T1 T5 T7 T11 T15 T9 T14 
Iteracao 2:

Caminhos para o vertice 1.
	0 - 1
	0 - 2 - 1
	0 - 4 - 3 - 1
Caminhos para o vertice 2.
	0 - 2
	0 - 1 - 2
Caminhos para o vertice 3.
	0 - 1 - 3
	0 - 4 - 3
Caminhos para o vertice 4.
	0 - 4
	0 - 1 - 3 - 4
Caminhos para o vertice 5.
	0 - 4 - 5
	0 - 1 - 2 - 6 - 5
	0 - 1 - 8 - 7 - 5
	0 - 1 - 3 - 4 - 5
Caminhos para o vertice 6.
	0 - 2 - 6
	0 - 1 - 2 - 6
Caminhos para o vertice 7.
	0 - 1 - 8 - 7
	0 - 2 - 1 - 8 - 7
Caminhos para o vertice 8.
	0 - 1 - 8
	0 - 2 - 1 - 8
	0 - 4 - 3 - 1 - 8
Caminhos para o vertice 9.
	0 - 1 - 2 - 6 - 9
Caminhos para o vertice 10.
	0 - 1 - 8 - 10
	0 - 2 - 1 - 8 - 10
Caminhos para o vertice 11.
	0 - 4 - 11
	0 - 1 - 3 - 4 - 11
Caminhos para o vertice 12.
	0 - 4 - 11 - 12
Caminhos para o vertice 13.
	0 - 1 - 2 - 6 - 13
Caminhos para o vertice 14.
	0 - 4 - 11 - 14
Caminhos para o vertice 15.

T9 T5 T11 T7 T1 T3 T6 T4 T8 T0 T10 T13 T15 T12 T14 T2 
Iteracao 3:

Caminhos para o vertice 1.
	0 - 1
	0 - 2 - 1
	0 - 4 - 3 - 1
Caminhos para o vertice 2.
	0 - 2
	0 - 1 - 2
Caminhos para o vertice 3.
	0 - 1 - 3
	0 - 4 - 3
Caminhos para o vertice 4.
	0 - 4
	0 - 1 - 3 - 4
Caminhos para o vertice 5.
	0 - 4 - 5
	0 - 1 - 2 - 6 - 5
	0 - 1 - 8 - 7 - 5
	0 - 1 - 3 - 4 - 5
Caminhos para o vertice 6.
	0 - 2 - 6
	0 - 1 - 2 - 6
Caminhos para o vertice 7.
	0 - 1 - 8 - 7
	0 - 2 - 1 - 8 - 7
Caminhos para o vertice 8.
	0 - 1 - 8
	0 - 2 - 1 - 8
	0 - 4 - 3 - 1 - 8
Caminhos para o vertice 9.
	0 - 2 - 6 - 9
	0 - 1 - 2 - 6 - 9
Caminhos para o vertice 10.
	0 - 1 - 8 - 10
	0 - 2 - 1 - 8 - 10
Caminhos para o vertice 11.
	0 - 4 - 11
	0 - 1 - 3 - 4 - 11
Caminhos para o vertice 12.
	0 - 4 - 11 - 12
	0 - 1 - 8 - 10 - 12
Caminhos para o vertice 13.
	0 - 1 - 2 - 6 - 13
	0 - 4 - 11 - 14 - 13
Caminhos para o vertice 14.
	0 - 4 - 11 - 14
	0 - 1 - 8 - 10 - 14
	0 - 1 - 2 - 6 - 13 - 14
Caminhos para o vertice 15.
	0 - 4 - 11 - 12 - 15
	0 - 4 - 11 - 14 - 15

T11 T15 T3 T9 T12 T7 T14 T13 T5 T10 T1 T6 T4 T8 T2 T0 

Saida:
Realizadas 3 iteracoes.

Caminhos para o vertice 1.
	0 - 1
	0 - 2 - 1
	0 - 4 - 3 - 1
Caminhos para o vertice 2.
	0 - 2
	0 - 1 - 2
Caminhos para o vertice 3.
	0 - 1 - 3
	0 - 4 - 3
Caminhos para o vertice 4.
	0 - 4
	0 - 1 - 3 - 4
Caminhos para o vertice 5.
	0 - 4 - 5
	0 - 1 - 2 - 6 - 5
	0 - 1 - 8 - 7 - 5
	0 - 1 - 3 - 4 - 5
Caminhos para o vertice 6.
	0 - 2 - 6
	0 - 1 - 2 - 6
Caminhos para o vertice 7.
	0 - 1 - 8 - 7
	0 - 2 - 1 - 8 - 7
Caminhos para o vertice 8.
	0 - 1 - 8
	0 - 2 - 1 - 8
	0 - 4 - 3 - 1 - 8
Caminhos para o vertice 9.
	0 - 2 - 6 - 9
	0 - 1 - 2 - 6 - 9
Caminhos para o vertice 10.
	0 - 1 - 8 - 10
	0 - 2 - 1 - 8 - 10
Caminhos para o vertice 11.
	0 - 4 - 11
	0 - 1 - 3 - 4 - 11
Caminhos para o vertice 12.
	0 - 4 - 11 - 12
	0 - 1 - 8 - 10 - 12
Caminhos para o vertice 13.
	0 - 1 - 2 - 6 - 13
	0 - 4 - 11 - 14 - 13
Caminhos para o vertice 14.
	0 - 4 - 11 - 14
	0 - 1 - 8 - 10 - 14
	0 - 1 - 2 - 6 - 13 - 14
Caminhos para o vertice 15.
	0 - 4 - 11 - 12 - 15
	0 - 4 - 11 - 14 - 15
\end{lstlisting}

\subsection{Compilando}
Para compilar o EP para Sistemas Operacionais Unix, basta executar o comando "cmake . \&\&  make", que rodará o CMake incluso com o EP, gerando assim um arquivo Makefile, que então será executado, gerando o executável.

\end{document}
